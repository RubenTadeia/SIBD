% -*- root: main.tex -*-

\section{Análise do Filtro}

\subsection{Resposta em Frequência}

\subsubsection*{Alínea (a)}

O circuito de \emph{Sallen and Key} pode ser interpretado como um Ampop realimentado positivamente por uma rede RC. De acordo com a figura \ref{fig:sallen} e aplicando o teorema da sobreposição, a tensão $V_1$ pode ser escrita da forma
\begin{IEEEeqnarray}{lCr}\label{eq:V1dep}
V_1 = T_{12}V_2 + T_{13}V_3,
\end{IEEEeqnarray}
\noindent onde $T_{12}$ e $T_{13}$ são, respectivamente, as funções de transferência de $V_1$ em relação a $V_2$ ($V_3=0$) e $V_3$ ($V_2=0$).

\begin{figure}[!ht]
\centering
\includegraphics[scale=0.9]{img/sallen_key.pdf}
\caption{Circuito \emph{Sallen and Key}.}
\label{fig:sallen}
\end{figure}

Para $T_{12}$, o circuito correspondente encontra-se representado na figura \ref{fig:rc1}.
\begin{figure}[!ht]
\centering
\includegraphics[scale=0.9]{img/T12.pdf}
\caption{Rede RC para análise da função $T_{12}$.}
\label{fig:rc1}
\end{figure}

Considerando a análise do circuito utilizando as impedências $Z_{Cx}$ e $R_x$ ($x=1,2$), podem ser estabelecidas as seguintes equações:
\begin{IEEEeqnarray}{lCr}
I_2 = -\frac{V_1}{Z_{C2}},\\
V_x = V_1\cdot\left(1+\frac{R_2}{Z_{C2}}\right),\\
I_{C1} = \frac{V_x}{Z_{C1}},\\
I_{R1} = \frac{V_x-V_2}{R_1}.
\end{IEEEeqnarray}

A aplicação de KCL ao circuito, \emph{i.e.}, $I_2 = I_{C1} + I_{R1}$ permite a obtenção da função de transferência $T_{12}$,
\begin{IEEEeqnarray}{lCr}\label{eq:Tgeral}
T_{12} = \displaystyle\frac{Z_{C1}\cdot Z_{C2}}{R_1\cdot R_2+R_1\cdot Z_{C1}+R_1\cdot Z_{C2}+ R_2\cdot Z_{C1}+Z_{C1}\cdot Z_{C2}}
\end{IEEEeqnarray}

Substituído $Z_{Cx}$ com a igualdade patente na Fig. \ref{fig:sallen}, obtém-se
\begin{IEEEeqnarray}{lCr}
T_{12} = \displaystyle\frac{\frac{1}{C_1 C_2 R_1 R_2}}{s^2+\left(\frac{1}{C_2 R_2}+\frac{1}{C_1 R_2}+\frac{1}{C_1 R_1}\right)s+\frac{1}{C_1 C_2 R_1 R_2} }.
\end{IEEEeqnarray}

A função foi obtida utilizando o programa MATLAB, cujo código está disponível em anexo.

A Fig. \ref{fig:rc2} representa o circuito corresponde à análise da função de transferência $T_{13}$.
\begin{figure}[!ht]
\centering
\includegraphics[scale=0.9]{img/T13.pdf}
\caption{Rede RC para análise da função $T_{13}$.}
\label{fig:rc2}
\end{figure}


\noindent Comparando com a Fig. \ref{fig:rc1}, observa-se que apenas $R_1$ está trocado com $C_1$. Como tal, basta apenas substituir na equação \ref{eq:Tgeral} $Z_{C1}\leftrightarrow R_1$. Assim, obtém-se:
\begin{IEEEeqnarray}{lCr}
T_{13} = \displaystyle\frac{s\frac{1}{C_2 R_2}}{s^2+\left(\frac{1}{C_2 R_2}+\frac{1}{C_1 R_2}+\frac{1}{C_1 R_1}\right)s+\frac{1}{C_1 C_2 R_1 R_2} }.
\end{IEEEeqnarray}

\noindent A equação anterior foi obtida com recurso ao MATLAB, cujo código está disponível em anexo. Veja-se que, como a rede RC é linear, os denominadores de $T_{12}$ e $T_{13}$ são iguais.

Considerando o ampop como ideal, com ganho $A$ infinito, a tensão $V_3$ pode ser descrita em termos de $V_1$ como um sistema com realimentação negativa através do jogo de resistências $r_1$ e $r_2$, cuja equação é dada por
\begin{IEEEeqnarray}{lCr}\label{eq:V3dep}
V_3 = kV_1, \quad k= \frac{A}{1+A\beta}
\end{IEEEeqnarray}
\noindent em que  $\beta$  é dado pela relação de resistências
\begin{IEEEeqnarray}{lCr}
\beta = \frac{1}{1+r_1/r_2}.
\end{IEEEeqnarray}

Ora tendo $A \rightarrow \infty $, segue que $k\approx1/\beta$. Posto isto, juntando as equações \ref{eq:V1dep} e \ref{eq:V3dep}, obtém-se a função de transferência do circuito, dada por:
 \begin{IEEEeqnarray}{lCr}\label{eq:Tp}
T(s) = \frac{T_{12}(s)k}{1-T_{13}(s)k}.
\end{IEEEeqnarray}

Ora, como se pretendende ganho unitário, tal só é possível se $k=1$. Como tal, tem que se ter $r_1=\infty$ e, por simplicidade, $r_2=0$.

\subsubsection*{Alínea (b)}

Pelo desenvolvimento da alínea (a) conclui-se que a função de transferência do sistema é reescrita sob a forma:
\begin{IEEEeqnarray}{lCr}\label{eq:Trans}
T(s) = \displaystyle\frac{\frac{1}{C_1 C_2 R_1 R_2}}{s^2+\left(\frac{R_1+R_2}{C_1 R_1 R_2}\right)s+\frac{1}{C_1 C_2 R_1 R_2} }.
\end{IEEEeqnarray}

Os parâmetros $T_0$, $f_p$, $Q_p$ e $K$ retiram-se, imediatamente, da equação \ref{eq:Trans}, obtendo-se os resultados da tabela \ref{tab:param}.
\begin{table}[!ht]
\centering
\begin{tabular}{|c|c|c|}
\hline
\textbf{Parâmetro} & \textbf{Equação Simbólica} &\textbf{ Valor} \\
\hline
$K$ & $\frac{1}{C_1 C_2 R_1 R_2}$ & 40404040.404 \\[1ex]
\hline
$w_p$ &  $\frac{1}{\sqrt{C_1 C_2 R_1 R_2}}$ &  6.3564 $kHz$ \\[1ex]
\hline
$f_p$ & $\frac{1}{2\pi\sqrt{C_1 C_2 R_1 R_2}}$ &  1011.6552 $Hz$\\[1ex]
\hline
$Q_p$ &  $\frac{1}{R_1+R_2}\sqrt{\frac{C_1 R_1 R_2}{C_2}}$ & 4.7673\\[1ex]
\hline
$T_0$ &$\frac{K}{w_p^2}$ & 1\\[1ex]
\hline
\end{tabular}
\caption{Equações simbólicas e valores numéricos dos parâmetros $T_0$, $f_p$, $Q_p$, $K$ e $w_p$.}
\label{tab:param}
\end{table}

\subsubsection*{Alínea (c)}

Tratando-se de um filtro passa-baixo de ganho unitário a baixas frequências, a largura de banda é determinada pela frequência cujo módulo, corresponde a $-3 \ dB$. Em unidades lineares verifica-se
\begin{IEEEeqnarray}{lCr}\label{eq:bnd}
|T(jw_b)| = \displaystyle\frac{1}{\sqrt{2}}.
\end{IEEEeqnarray}

Genericamente, o módulo da função $T(s)$ pode ser escrito sob a forma
\begin{IEEEeqnarray}{lCr}\label{eq:genabs}
|T(jw)| = \displaystyle\frac{|K|}{\sqrt{(w_p^2-w^2)^2+\left(\frac{w_p}{Q_p}w\right)^2}}.
\end{IEEEeqnarray}

Aplicando a equação \ref{eq:genabs} em \ref{eq:bnd} e resolvendo a igualdade em MATLAB, dividindo o resultado por $2\pi$, para obter a frequência em $Hz$, verifica-se que a largura de banda é $b=1559.6376 \ Hz$.

\subsubsection*{Alínea (d)}

Os diagramas de amplitude e fase encontram-se na Figura \ref{fig:bode}. Estes foram foi obtidos utilizando o programa MATLAB, cujo código está disponível em anexo.

\begin{figure}[!thpb]
\centering
\includegraphics[width=\textwidth]{img/bode.png}
\caption{Diagramas de amplitude e fase do filtro passa-baixo.}
\label{fig:bode}
\end{figure}

\subsubsection*{Alínea (e)}

De modo a calcular experimentalmente os parâmetros $f_p$, $Q_p$, $b$ e $G_0$ podem ser efectuados os seguintes passos. Note-se que, devido ao \emph{slew-rate}, a tensão de entrada deverá de ser enquadrada com a frequência de operação.

\paragraph{Passo 1 - Cálculo de $G_0$}

Introduzir um sinal com $100\ Hz$ de frequência, e calcular o ganho. O ganho obtido corresponde a $G_0$.

\paragraph{Passo 2 - Cálculo de $b$}

Variar a frequência do sinal de entrada até obter uma tensão de saída igual a $G_0 \cdot V_{in}/\sqrt{2}$. A frequência do sinal de entrada corresponde à largura de banda.

\paragraph{Passo 3 - Cálculo de $Q_p$}

Variar a frequência do sinal de entrada até obter ganho máximo à saída. O valor numérico do ganho resultante corresponde a uma aproximação de $Q_p$. Tal aproximação é válida apenas quando $0.25\cdot Q_p^{-2} \ll 1$, como se verifica pela expressão \ref{eq:tmax_qp}.%
%
\begin{IEEEeqnarray}{lCr}\label{eq:tmax_qp}
|T|_{max} = \displaystyle\frac{Q_p}{\sqrt{1- 0.25 \cdot Q_p^{-2}}}
\end{IEEEeqnarray}

\pagebreak
No circuito em questão tem-se $|T|_{max} \approx 1 / 0.9945 \cdot Q_p$. De facto, para pequenas variações de $Q_p$ em torno do seu valor esperado, a dependência linear anterior tem uma variância ainda menor, pelo que, para obter um valor mais exacto de $Q_p$ pode assumir-se%
%
\begin{IEEEeqnarray}{lCr}\label{eq:qp_from_tmax}
Q_p \approx 0.9945 \cdot |T|_{max}\ .
\end{IEEEeqnarray}

\paragraph{Passo 4 - Cálculo de $f_p$} Com o mesmo sinal de entrada do passo anterior, verifica-se a equação%
%
\begin{IEEEeqnarray}{lCr}
w_{max} = \displaystyle w_p \cdot {\sqrt{1- 0.5 \cdot Q_p^{-2}}} \ .
\end{IEEEeqnarray}

Dado o alto valor de $Q_p$, poder-se-ia simplesmente tomar $f_p$ pelo valor numérico de $f_{max}$, mas optou-se por efectuar o mesmo procedimento descrito no passo 3. Para o valor esperado de $Q_p$ e para pequenas variações em torno deste ponto, verifica-se a aproximação
%
\begin{IEEEeqnarray}{lCr}\label{eq:fp_from_fmax}
f_p \approx 1.0112 \cdot f_{max}\ .
\end{IEEEeqnarray}
