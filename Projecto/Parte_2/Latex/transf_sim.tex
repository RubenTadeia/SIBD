% % -*- root: main.tex -*-

\section{Transformação do Filtro e Simulação}

\subsubsection*{Simulação do passa-baixo}

Na Figura \ref{fig:sim_lp} encontra-se a simulação do filtro passa-baixo, onde se utilizou o amplificador operacional RH37C, cujo produto GB tem valor 2 GHz.

\begin{figure}[!thpb]
\centering
\includegraphics[width=0.95\textwidth, trim={0 0.6cm 0 0},clip]{img/lp_freq_ltspice.pdf}
\caption{Simulação da resposta em frequência do passa-baixo.}
\label{fig:sim_lp}
\end{figure}


Por leitura do ganho máximo e aplicação da expressão \ref{eq:qp_from_tmax}, determina-se $Q_p = 4.7662$. Pela expressão \ref{eq:fp_from_fmax}, determina-se $f_p = 1012.3$ Hz. Estes valores são corroborados pela Tabela \ref{tab:param}, o que comprova a validade da simulação e das aproximações efectuadas na obtenção do polo através do máximo do ganho. É de notar que o produto GB não finito do amplificador operacional introduz erros, mas com $GB = 2 \pi \cdot 2$ GHz estes serão 20 vezes inferiores aos indicados na alínea (c) de 2.2, pelo que foram ignorados.

Verifica-se ainda que a largura de banda foi correctamente prevista.

\subsubsection*{Alínea (a)}

Pela transformação RC-CR, é obtido um filtro passa-alto a partir do filtro passa-baixo. Tal pode ser explicado pelos passos seguidos na transformação.

O primeiro passo consiste em normalizar o filtro passa-baixo, de forma a obter um protótipo $T_{LP}(S)$ pela equação%
%
\begin{IEEEeqnarray}{lCr}
T_{LP}(s_{L}) \xrightarrow[S = \displaystyle\frac{s_{L}}{w_{PL}}]{} T_{LP}(S)\ ,
\end{IEEEeqnarray}%
%
onde $w_{PL}$ é o limiar da banda de passagem do passa-baixo.

O segundo passo da tranformação é a desnormalização do protótipo, transformando-o num passa-alto, pela equação%
%
\begin{IEEEeqnarray}{lCr}
T_{LP}(S) \xrightarrow[s_H = \displaystyle\frac{w_{PH}}{S}]{} T_{HP}(s_H)\ ,
\end{IEEEeqnarray}%
%
onde $w_{PH}$ é o limiar da banda de passagem do passa-alto.

Assim, pode-se definir uma função genérica do filtro passa-alto resultante através da expressão \ref{eq:hp_transf}, onde $Q_0$ e $w_0$ são a frequência e factor de qualidade do polo do filtro passa-baixo.%
%
\begin{IEEEeqnarray}{lCr}\label{eq:hp_transf}
	T_{HP}(s) = T_{LP}(s)\bigg|_{\displaystyle s=\frac{w_{PL}\cdot w_{PH}}{s}}
			= \frac{s^2}{\displaystyle s^2 + s \cdot \frac{w_{PL}\cdot w_{PH} / w_0}{Q_0} + \left( \frac{w_{PL}\cdot w_{PH}}{w_0} \right)^2}
\end{IEEEeqnarray}%

\begin{table}[!ht]
\flushright
\begin{tabular}{|ll|ll|lr}
$R$ & \multirow{4}{*}{$\xrightarrow[]{\displaystyle s = \frac{w_{PL}\cdot w_{PH}}{s}}$}
	& R & \multirow{4}{*}{$\xrightarrow[]{\displaystyle \hat{Z} = \frac{Z}{\rho s}}$} &
	$\refstepcounter{equation}\label{eq:r_transf}\displaystyle\frac{R}{\rho s} = \frac{1}{\hat{C} s}, \quad \hat{C} = \frac{\rho}{R} $ & \quad \quad \quad \quad (\theequation)\\

&&&&& \\ \cline{1-1} \cline{3-3} \cline{5-5} &&&&& \\

$\displaystyle\frac{1}{sC}$ & & $\displaystyle\frac{s}{w_{PL} \cdot w_{PH} \cdot C}$ & &
		$\refstepcounter{equation}\label{eq:c_transf}\displaystyle\frac{1}{w_{PL} \cdot w_{PH} \cdot C \rho} = \hat{R}$ & (\theequation)
\end{tabular}
\caption*{}
\end{table}

\pagebreak
Pelas transformações dos componentes passivos descritas em \ref{eq:r_transf} e \ref{eq:c_transf}, é possível estabelecer uma relação entre os parâmetros do filtro e os valores dos componentes a utilizar no novo filtro. Destas duas esquações vem que%
%
\begin{IEEEeqnarray}{lCr}\label{eq:hats}
	\displaystyle w_{PL} \cdot w_{PH} = \frac{1}{R \hat{R} C \hat{C}} \ .
\end{IEEEeqnarray}%

Com base em \ref{eq:hp_transf} e \ref{eq:hats}, a frequência do polo do novo filtro, $f_1$, pode ser definida em função dos valores componentes passivos antes e depois da transformação e da frequência do polo do passa-baixo, $f_0$.
%
\begin{IEEEeqnarray}{lCr}
	f_1 = \frac{1}{R \hat{R} C \hat{C} f_0}
\end{IEEEeqnarray}%

O factor de qualidade do polo não sofre alteração, bem como o parâmetro $K$ que é definido pela relação entre $r_1$ e $r_2$. A tabela \ref{tab:transf_sum} sumariza a transformação dos parâmetros do filtro.

\begin{table}[!ht]
	\centering
	\begin{tabular}{|c|c|c|c|}
		\hline
		$Topologia$ & $f_p$ & $Q_p$ & $K$ \\
		\hline \hline
		LP & $f_0$ & $Q_0$ & $K_0$ \\
		\hline
		HP & $ \frac{1}{R \hat{R} C \hat{C} f_0}$ & $Q_0$ & $K_0$ \\ \hline
	\end{tabular}
	\caption{Sumário da transformação dos parâmetros do filtro}
	\label{tab:transf_sum}
\end{table}

\subsubsection*{Alínea (b)}

A selecção dos componentes foi feita de acordo com o que é sugerido pela disposição da placa de ensaios, isto é, $C_1 \rightarrow \hat{R_1}$ e $C_2 \rightarrow \hat{R_2}$. O esquema do circuito resultante está apresentado na figura \ref{fig:hp_transf_circ} e na figura \ref{fig:sim_hp} apresenta-se a simulação da resposta em frequência do novo filtro.

\begin{figure}[!thpb]
\centering
\includegraphics[width=0.5\textwidth]{img/sallen_key_hp.pdf}
\caption{Circuito eléctrico do passa-alto transformado.}
\label{fig:hp_transf_circ}
\end{figure}

\begin{figure}[!thpb]
\centering
\includegraphics[width=0.95\textwidth, trim={0 0.6cm 0 0},clip]{img/hp_freq_ltspice.pdf}
\caption{Simulação da resposta em frequência do passa-alto transformado.}
\label{fig:sim_hp}
\end{figure}

Por observação da resposta em frequência retira-se directamente que o valor de $w_{PH}$ é $2\pi \cdot 687.99$ Hz. Ora, tendo em conta o valor de $w_{PL}$ (determinado na alínea (c) da Secção 2.2) e expressão \ref{eq:hats}, o valor esperado é $2\pi \cdot 721.83$ Hz (utilizando o par de valores $C_1 \hat{R_1}$) ou $2\pi \cdot 656.21$ Hz (utilizando o par de valores $C_2 \hat{R_2}$). O valor foi correctamente previsto tendo em conta que tal discrepância deve-se ao facto dos componentes passivos seguirem valores tabelados. De forma a calcular o valor exacto teórico, uma análise detalhada semelhante à da Secção 2.1 seria necessária, o que está fora do âmbito deste trabalho.

Verifica-se ainda que a frequência do máximo e o valor do máximo é semelhante no filtro original e transformado, o que é suportado pelo facto do factor de qualidade e frequência do polo não se alterar. Mais uma vez, o valor dos componentes passivos usadas introduz apequenas variâncias.

\subsubsection*{Alínea (c)}

O filtro obtido por aplicação da transformação complementar está representado na figura \ref{fig:bp_transf_circ}, através do seu esquema eléctrico e a sua resposta em frequência está representada na figura \ref{fig:sim_bp}. Por observação da resposta em frequência verifica-se que se trata de um passa-banda.

\begin{figure}[!thpb]
\centering
\includegraphics[width=0.5\textwidth]{img/sallen_key_bp.pdf}
\caption{Circuito eléctrico do passa-banda complementar do passa-baixo inicial.}
\label{fig:bp_transf_circ}
\end{figure}

\begin{figure}[!thpb]
\centering
\includegraphics[width=0.95\textwidth, trim={0 0.6cm 0 0},clip]{img/bp_freq_ltspice.pdf}
\caption{Simulação da resposta em frequência do passa-banda complementar do passa-baixo inicial.}
\label{fig:sim_bp}
\end{figure}


\subsubsection*{Alínea (d)}

Os polos de uma montagem com um único amplificador operacional não se alteram quando o seus terminais de entrada e também a saída e terminal de massa são permutados. Com base neste teorema, conclui-se que, na transformação complementar, os pólos do filtro passa-banda são os mesmos que os pólos do passa-baixo.

Assim, os polos do novo filtro são descritos pelas mesmas expressões calculadas na análise inicial, pelo que as sensibilidades e, consequentemente, os desvios serão os mesmos. De facto, observa-se pela figura \ref{fig:sim_bp} que a frequência do máximo do ganho corresponde à frequência dos polos do passa-baixo.

Mais, pode-se calcular facilmente a equação que rege a função de transferência deste circuito. Repara-se que a forma do sistema em causa (em termos de rede RC conectada a um AmpOp) é muito semelhante ao apresentado na figura \ref{fig:sallen}, mas neste caso apresenta-se realimentação negativa ($-A$). Contudo, o valor de $k$ mantém-se, $k'=1$.

Fazendo um paralelo com a figura \ref{fig:sallen} e a equação \ref{eq:Tp}, o novo circuito é definido por
 \begin{IEEEeqnarray}{lCr}\label{eq:Tn}
T_N(s) = \frac{T_{1'2'}(s)k'}{1-T_{1'3'}(s)k'}.
\end{IEEEeqnarray}
\noindent Simplificando a expressão, verifica-se que
 \begin{IEEEeqnarray}{lCr}
T_N(s) = \frac{N_{1'2'}(s)}{D(s)-N_{1'3'}(s)}.
\end{IEEEeqnarray}

Ora sabe-se pelo teorema referido no primeiro parágrafo da alínea que o denominador de $T_N(s)$ é igual ao da função de transferência do passa-baixo. Resta pois, calcular o valor de $N_{1'2'}(s)$. Curiosamente, o circuito resultante é idêntico ao calculado pela função $T_{13}(s)$ - ver Fig. \ref{fig:rc2} - pelo que
 \begin{IEEEeqnarray}{lCr}
N_{1'2'}(s) = \frac{s}{C_2R_2}.
\end{IEEEeqnarray}

Concluindo, após recurso ao MATLAB (código em anexo), conclui-se que a nova função de transferência equaciona em
 \begin{IEEEeqnarray}{lCr}
T_N(s) = \displaystyle\frac{\frac{s}{C_2R_2}}{s^2+\left(\frac{R_1+R_2}{C_1 R_1 R_2}\right)s+\frac{1}{C_1 C_2 R_1 R_2} }.
\end{IEEEeqnarray}

O gráfico resultante da função $T_N(s)$, presente na Fig. \ref{fig:Tn}, em tudo assemelha-se à da Fig. \ref{fig:sim_bp}, comprovando, assim, o teorema indicado.

\begin{figure}[!thpb]
\centering
\includegraphics[width=\textwidth]{img/bode_bp.png}
\caption{Simulação teórica da resposta em frequência do passa-banda complementar do passa-baixo inicial.}
\label{fig:Tn}
\end{figure}
