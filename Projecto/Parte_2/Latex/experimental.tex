% -*- root: main.tex -*-

\section{Procedimento Experimental}

De forma a evitar problemas de medição dos valores experimentais devido a erros provocados pela limitação do slew-rate do amplificador operacional, efectuou-se a análise da tensão máxima à saída permitida, em função da frequência do sinal. Tal análise foi feita utilizando MATLAB, para geração do gráfico, tendo sido o código colocado em anexo. O gráfico destes valores máximos está apresentado na figura \ref{fig:slew_rate}, tendo-se assumido um slew-rate de 0.5 V/$\mu$s, tal como indicado no valor típico do datasheet.

\begin{figure}[!thpb]
\centering
\includegraphics[width=0.9\textwidth]{img/slew_rate.png}
\caption{Tensão de máxima de saída em função da frequência do sinal devido a um slew-rate de 0.5 V/$\mu$s.}
\label{fig:slew_rate}
\end{figure}

A partir do gráfico da tensão de saída máxima, optou-se por dimensionar as tensões de entrada nos vários procedimentos experimentais de forma a ter menos de 5 V à saída, o que garante que não se verifica as limitações do slew-rate a nenhuma independentemente da frequência.

\subsection{Caracterização do Amplificador Operacional}

\subsubsection*{Alínea (a)}

O módulo utilizado tem número \quad\quad\quad.

\subsubsection*{Alínea (b)}

Com um sinal de entrada de amplitude 5 V e frequência 100 Hz, a reposta do amplificador operacional está representada em anexo, na página \quad\quad\quad.

% Com um sinal de entrada de amplitude 5 V e frequência 100 Hz, a reposta do amplificador operacional está representada em anexo, na figura \ref{fig:buffer_resp}.

% \begin{figure}[!thpb]
% \centering
% \includegraphics[width=0.5\textwidth]{img/plot.png}
% \caption{Resposta do amplificador operacional a um sinal de amplitude 8 V e frequência 100 Hz na montagem de ganho unitário.}
% \label{fig:buffer_resp}
% \end{figure}

A baixa frequência o ganho do amplificador operacional é elevado o suficiente para que se comporte como um amplificar ideal, pelo que, com o feedback unitário, o sinal de saída é igual ao de entrada.

\subsubsection*{Alínea (c)}

A resposta do amplificador operacional na montagem com ganho unitário e a funcionar com um ganho 3 dB abaixo do ganho DC está representada em anexo, na página \quad\quad\quad.

% A resposta do amplificador operacional na montagem com ganho unitário e a funcionar com um ganho 3 dB abaixo do ganho DC está representada na figura \ref{fig:unit_corte}.

% \begin{figure}[!thpb]
% \centering
% \includegraphics[width=0.5\textwidth]{img/plot.png}
% \caption{Resposta do amplificador operacional a um sinal de amplitude 5 V e frequência \colorbox{red}{FREQ!!} na montagem de ganho unitário.}
% \label{fig:unit_corte}
% \end{figure}

A muito baixa frequência, o ganho obtido foi de \quad\quad\quad. Assim, nesta situação o ganho alvo foi de \quad\quad\quad. Com uma amplitude de entrada de \quad\quad\quad e amplitude de saída de \quad\quad\quad, o ganho à frequência \quad\quad\quad é de \quad\quad\quad. Conclui-se que o produto GB é \quad\quad\quad.

\subsection{Caracterização do Filtro}

\subsubsection*{Alínea (b)}

Os parâmetros $G_0$, $f_p$, $Q_p$ e $b$ calculados experimentalmente estão apresentados na Tabela \ref{tab:param_exp}, bem como as características do sinal de entrada, amplitude do sinal de saída e indicação da página em que o plot do representação dos sinais está anexada.

\begin{table}[!ht]
\centering
\begin{tabular}{|c|c|c|c|c|c|}
\hline
{Parâmetro} &{ Valor} & $V_i$ [mV] & $V_o$ [mV] & $f_{test}$ [Hz] & Pág. anexo \\
\hline\hline
$b$ & \quad\quad\quad\quad\quad\quad\quad & \quad\quad\quad\quad\quad\quad\quad &
	\quad\quad\quad\quad\quad\quad\quad & \quad\quad\quad\quad\quad\quad\quad & \\
\hline
$f_p$ &&&&& \\
\hline
$Q_p$ &&&&& \\
\hline
$G_0$ &&&&& \\
\hline
\end{tabular}
\caption{Valores numéricos experimentais dos parâmetros $G_0$, $f_p$, $Q_p$ e $b$.}
\label{tab:param_exp}
\end{table}

\subsubsection*{Alínea (c)}

De forma a fazer uma análise mais detalhada, efectuou-se a caracterização da resposta do filtro em quatro pontos por década, igualmente espaçados numa escala logarítmica, o que explica os valores de frequência usados. Os resultados obtidos estão apresentados na Tabela \ref{tab:freqs_medidas}.

Os pontos obtidos foram marcados nos diagramas de Bode, na Figura \ref{fig:bode}. Adicionalmente, anexa-se na página \quad\quad\quad a resposta em frequência obtida através do analizador de espectros.

\begin{table}[!ht]
\centering
\begin{tabular}{|c|c|c|c|c|c|}
\hline
$f_{test}$ [Hz] & $V_i$ [mV] & $V_o$ [mV] & Ganho [dB] & Fase [º] \\
\hline\hline
100 & \quad\quad\quad\quad\quad\quad\quad & \quad\quad\quad\quad\quad\quad\quad
	& \quad\quad\quad\quad\quad\quad\quad & \quad\quad\quad\quad\quad\quad\quad \\
\hline
178 &&&& \\
\hline
316 &&&& \\
\hline
562 &&&& \\
\hline
1000 &&&& \\
\hline
1780 &&&& \\
\hline
3160 &&&& \\
\hline
5620 &&&& \\
\hline
10000 &&&& \\
\hline
\end{tabular}
\caption{Pontos experimentais medidos para caracterização da resposta em frequência do filtro.}
\label{tab:freqs_medidas}
\end{table}

\subsubsection*{Alínea (d) - Comentários}

\begin{table}[!hb]
\begin{tabularx}{\textwidth}{X}
\hline \\ \hline \\ \hline \\ \hline \\ \hline \\ \hline \\ \hline \\ \hline \\ \hline \\ \hline \\
\hline \\ \hline \\ \hline \\ \hline \\ \hline \\ \hline \\ \hline \\ \hline \\ \hline \\ \hline \\ \hline \\ \hline \\
\end{tabularx}
\end{table}

\subsection{Análise de Sensibilidades}

\subsubsection*{Alínea (a)}

Os valores calculados dos erros relativos de $f_p$ e $Q_p$, com a instrodução de desvios nos componentes passivos, estão apresentados na Tabela \ref{tab:param_exp_desvios_pass}.

\begin{table}[!ht]
\centering
\begin{tabular}{|c|c|c|c|}
\hline
{Parâmetro} &{ Valor} & Desvio esperado [\%] & Desvio medido [\%] \\
\hline\hline
$f_p$ & \quad\quad\quad\quad\quad\quad\quad & $-10$ & \quad\quad\quad\quad\quad\quad\quad  \\
\hline
$Q_p$ && $+10$ & \\
\hline
\end{tabular}
\caption{Desvios relativos de $f_p$ e $Q_p$ medidos ao introduzir desvios nos componentes passivos.}
\label{tab:param_exp_desvios_pass}
\end{table}

\subsubsection*{Alínea (b)}

Os valores calculados dos erros relativos de $f_p$ e $Q_p$, com a instrodução de desvios no amplificador operacional, estão apresentados na Tabela \ref{tab:param_exp_desvios_act}.

\begin{table}[!ht]
\centering
\begin{tabular}{|c|c|c|c|}
\hline
{Parâmetro} &{ Valor} & Desvio esperado [\%] & Desvio medido [\%] \\
\hline\hline
$f_p$ & \quad\quad\quad\quad\quad\quad\quad & $-4.823$ & \quad\quad\quad\quad\quad\quad\quad  \\
\hline
$Q_p$ && $+4.823$ & \\
\hline
\end{tabular}
\caption{Desvios relativos de $f_p$ e $Q_p$ medidos ao introduzir desvios no amplificador operacional.}
\label{tab:param_exp_desvios_act}
\end{table}

\subsubsection*{Alínea (c)}

Os valores calculados dos erros relativos de $f_p$ e $Q_p$, com a instrodução de desvios no amplificador operacional e nos componentes passivos, estão apresentados na Tabela \ref{tab:param_exp_desvios_all}.

\begin{table}[!ht]
\centering
\begin{tabular}{|c|c|c|c|}
\hline
{Parâmetro} &{ Valor} & Desvio esperado [\%] & Desvio medido [\%] \\
\hline\hline
$f_p$ & \quad\quad\quad\quad\quad\quad\quad & $-14.823$ & \quad\quad\quad\quad\quad\quad\quad  \\
\hline
$Q_p$ && $+14.823$ & \\
\hline
\end{tabular}
\caption{Desvios relativos de $f_p$ e $Q_p$ medidos ao introduzir desvios no amplificador operacional e nos componentes passivos.}
\label{tab:param_exp_desvios_all}
\end{table}

\subsubsection*{Alínea (d) - Comentários}

\begin{table}[!hb]
\begin{tabularx}{\textwidth}{X}
\hline \\ \hline \\ \hline \\ \hline \\ \hline \\ \hline \\ \hline \\ \hline \\ \hline \\ \hline
\end{tabularx}
\end{table}

\begin{table}[!htpb]
\begin{tabularx}{\textwidth}{X}
\hline \\ \hline \\ \hline \\ \hline \\ \hline \\ \hline \\ \hline \\ \hline \\ \hline \\ \hline
\end{tabularx}
\end{table}
